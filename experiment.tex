%%%%%%%%%%%%%%%%%%%%%%%%%%%%%%%%%%%%%%%%%%%%%%%%%%%%%%%%%%%%%%%%%%%%%%%%%%%%%%%%%%%%
%Do not alter this block of commands.  If you're proficient at LaTeX, you may include additional packages, create macros, etc. immediately below this block of commands, but make sure to NOT alter the header, margin, and comment settings here. 
\documentclass[12 pt]{article}
\usepackage[margin=1in, top=1.25in, bottom=1.25in]{geometry} 
\usepackage{amsmath,amsthm,amssymb,amsfonts, enumitem, fancyhdr, color, comment, graphicx, environ, bm, tikz}
\usepackage[mathscr]{euscript}
\pagestyle{fancy}
\setlength{\headheight}{25pt}
\newenvironment{problem}[2][Problem]{\begin{trivlist}
\item[\hskip \labelsep {\bfseries #1}\hskip \labelsep {\bfseries #2.}]}{\end{trivlist}}
\newenvironment{sol}
    {\emph{Solution:}
    }
    {
    \qed
    }
\specialcomment{com}{ \color{blue} \textbf{Comment:} }{\color{black}} %for instructor comments while grading

\newtheorem{theorem}{Theorem}
\newtheorem*{corollary}{Corollary}
\newtheorem*{proposition}{Proposition}
\theoremstyle{definition}
\newtheorem*{definition}{Definition}

\NewEnviron{probscore}{\marginpar{ \color{blue} \tiny Problem Score: \BODY \color{black} }}
%%%%%%%%%%%%%%%%%%%%%%%%%%%%%%%%%%%%%%%%%%%%%%%%%%%%%%%%%%%%%%%%%%%%%%%%%%%%%%%%%



\newenvironment{amatrix}[1]{%
  \left[\begin{array}{@{}*{#1}{c}|c@{}}
}{%
  \end{array}\right]
}


%%%%%%%%%%%%%%%%%%%%%%%%%%%%%%%%%%%%%%%%%%%%%
%Fill in the appropriate information below
\fancyhf{}
\lhead{Justin Baum}  %replace with your name
\rhead{Math 544 \\ Spring 2019 \\ Homework 9} %replace XYZ with the homework course number, semester (e.g. ``Spring 2019"), and assignment number.
\lfoot{\thepage}
%%%%%%%%%%%%%%%%%%%%%%%%%%%%%%%%%%%%%%%%%%%%%


% The following are definitions for shortcuts for pieces of notation that I use a lot.
\newcommand{\N}{\mathbb{N}} % the natural numbers
\newcommand{\Z}{\mathbb{Z}} % the integers
\newcommand{\C}{\mathbb{C}} % the complex numbers 
\newcommand{\R}{\mathbb{R}} % the real numbers
\newcommand{\Q}{\mathbb{Q}} % the rational numbers
\newcommand{\F}{\mathbb{F}} % the Field
\renewcommand{\a}{\alpha}
\newcommand{\lin}[1]{\mathscr{L}(#1)}
\newcommand{\lgr}{\lambda}
\newcommand{\vv}[1]{\mathbf{v_{#1}}}
\newcommand{\vu}[1]{\mathbf{u_{#1}}}
\newcommand{\ve}[1]{\mathbf{#1}}
\newcommand{\poly}[2]{\mathscr{P}_#2(#1)}
\newcommand{\s}[1]{\mathscr{#1}}
\newcommand{\zero}{\textbf 0}
\newcommand{\vo}[1]{\mathbf{#1}}
\newcommand{\ddx}{\frac{d}{dx}}
\DeclareMathOperator{\rank}{rank}
\DeclareMathOperator{\range}{range}
\DeclareMathOperator{\nul}{null}
\DeclareMathOperator{\img}{img}


%%%%%%%%%%%%%%%%%%%%%%%%%%%%%%%%%%%%%%
%Do not alter this block.
\begin{document}
%%%%%%%%%%%%%%%%%%%%%%%%%%%%%%%%%%%%%%
\begin{problem}{3.5.12}
    a) Show that $\mathscr{B}=(1,x,\frac{3}{2}x^2-\frac{1}{2})$ is a basis of $\poly{\R}{2}$.\\
    b) Find the coordinate representation of $x^2$ with respect to $\mathscr{B}$.\\
    c) \item Let $D:\poly{\R}{2}\to\poly{\R}{2}$ be the derivative operator. Find the coordinate representation of D with respect to $\mathscr{B}$. \\
    d) \item Use your answers to the last two parts to calculate $\frac{d}{dx}(x^2)$.
\end{problem}
\begin{sol}
\begin{enumerate}[label=(\alph*)]
    \item Let $T: \poly{\R}{2}\to \R^3$ so $T(ax^2+bx+c)=\begin{bmatrix}
    a\\b\\c
    \end{bmatrix}$. Let $A=[T(\mathscr{B_1})|\dots|\mathscr{B_3}]$.
    \[A=\begin{bmatrix}
    0 & 0 & \frac{3}{2}\\
    0 & 1 & 0\\
    1 & 0 & \frac{-1}{2}
    \end{bmatrix}\]
    \newcommand{\vb}[1]{\vo{b_{#1}}}
    In RREF, $A=I_3$. Thus it is a basis.
    \item If $\mathscr{B}=(1,x,\frac{3}{2}x^2-\frac{1}{2})$, let $\vb{1}=1$, $\vb{2}=x$, $\vb{3}=\frac{3}{2}x^2-\frac{1}{2}$.\\
    Then $x^2=\frac{2}{3}(\frac{3}{2}x^2-\frac{1}{2})+\frac{1}{3}(1)=\frac{2}{3}\vb{3}+\frac{1}{3}\vb{1}$. So $x^2 = \begin{bmatrix}
    \frac{1}{3 }\\0\\\frac{2}{3}
    \end{bmatrix}_\s{B}$.
    \item Let $D=[\ddx\vb{1}|\ddx\vb{2}|\ddx\vb{3}]=\begin{bmatrix}
    0 & 0 & 0\\
    0 & 0 & 3\\
    0 & 1 & 0
    \end{bmatrix}_\s{B}$.
    \item If $x^2=\begin{bmatrix}\frac{1}{3}\\0\\\frac{2}{3}\end{bmatrix}_\s{B}$, then $D\begin{bmatrix}\frac{1}{3}\\0\\\frac{2}{3}\end{bmatrix}_\s{B}=\begin{bmatrix}
    0\\
    2\\
    0
    \end{bmatrix}_\s{B}$, which is $2\vb{2}=2x$. 
\end{enumerate}
\end{sol}

\begin{problem}{3.5.14a}
Let $R\in\lin{\R^2}$ be the counterclockwise 
\end{problem}
\end{document}
